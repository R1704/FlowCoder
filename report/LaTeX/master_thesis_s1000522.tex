\documentclass[twocolumn]{article}

\usepackage[colorlinks = true,
            linkcolor = black,
            urlcolor  = black,
            citecolor = blue,
            anchorcolor = blue]{hyperref}

\usepackage{booktabs} % For formal tables
\usepackage{url}
\usepackage{color}
\usepackage{enumitem}
\usepackage{todonotes}
\usepackage{float}
\hyphenation{Media-Eval}

% For making fixed-width aligned columns
\usepackage{array}
\newcommand{\PreserveBackslash}[1]{\let\temp=\\#1\let\\=\temp}
\newcolumntype{C}[1]{>{\PreserveBackslash\centering}p{#1}}
\newcolumntype{R}[1]{>{\PreserveBackslash\raggedleft}p{#1}}
\newcolumntype{L}[1]{>{\PreserveBackslash\raggedright}p{#1}}


% %%%%%%%%%%%%%%%%%%%%%%%%%%%%%%%%
% %% SET TITLE PAGE VALUES HERE %%
% %%%%%%%%%%%%%%%%%%%%%%%%%%%%%%%%
% %             ||               %
% %             ||               %
% %             \/               %

% \def\thesistitle{Using Network Relations for the Semantic Representation of Program Abstractions}
\def\thesistitle{A title}

\def\thesissubtitle{A project for the ages}


% %             /\               %
% %             ||               %
% %             ||               %
% %%%%%%%%%%%%%%%%%%%%%%%%%%%%%%%%
% %% SET TITLE PAGE VALUES HERE %%
% %%%%%%%%%%%%%%%%%%%%%%%%%%%%%%%%


%% FOR PDF METADATA
\title{\thesistitle}
\date{\thesisdate}


\begin{document}
\twocolumn[
  \begin{@twocolumnfalse}
    \begin{titlepage}
	\thispagestyle{empty}
	\newcommand{\HRule}{\rule{\linewidth}{0.5mm}}
	\center
	\textsc{\Large Radboud University Nijmegen}\\[.7cm]
	\includegraphics[width=25mm]{img/in_dei_nomine_feliciter.eps}\\[.5cm]
	\textsc{Faculty of Social Science}\\[0.5cm]
	
	\HRule \\[0.4cm]
	{ \huge \bfseries \thesistitle}\\[0.1cm]
	\textsc{\thesissubtitle}\\
	\HRule \\[.5cm]
	\textsc{\large Thesis MSc Artificial Intelligence}\\[.5cm]
	\clearpage
\end{titlepage}

    \end{@twocolumnfalse}
    \textbf{Student Information} \\

    \begin{tabular}{L{3cm}l}
                \emph{Surname:} & \textsc{Hommelsheim} \\
                \emph{First name:} & \textsc{Ron} \\
                \emph{Student number:} & \textsc{s1000522} \\
                \emph{E-mail address:} & \textsc{ron.hommelsheim@ru.nl} \\
                \emph{Course code:} & \textsc{SOW-MKI94 (45 EC)} \\
                \emph{AI Specialisation:} & \textsc{Cognitive Computing} \\
        \end{tabular}
	
  \vspace{1cm}
  \textbf{Supervisor Information} \\

    \begin{tabular}{L{3cm}l}
            \emph{Role:} & \textsc{Supervisor 1} \\
            \emph{Surname:} & \textsc{Thill} \\
            \emph{First name:} & \textsc{Serge} \\
            \emph{Institute:} & \textsc{Radboud University} \\
            \emph{E-mail address:} & \textsc{serge.thill@donders.ru.nl} \\
            \emph{Supervision type:} & \textsc{Internal} \\
            % internal / affiliated / external
    \end{tabular}

    \vspace{0.5cm}

    \begin{tabular}{L{3cm}l}
            \emph{Role:} & \textsc{Supervisor 2} \\
            \emph{Surname:} & \textsc{} \\
            \emph{First name:} & \textsc{} \\
            \emph{Institute:} & \textsc{Radboud University} \\
            \emph{E-mail address:} & \textsc{} \\
            \emph{Supervision type:} & \textsc{Internal} \\
            % internal / affiliated / external
    \end{tabular}

]
\clearpage


\begin{abstract}

\end{abstract}

\tableofcontents

\section{Introduction}
\section{Introduction}
Imagine your brain as an interactive game engine. Just as a game engine generates dynamic virtual environments, complete with rules and physics that players interact with, the brain constructs a model of the real world. This model includes rules (physical laws, social norms), entities (objects, people), and interactions (how things work and relate to each other). We learn to navigate and predict our environment, constantly updating our internal model based on new experiences and information.
This analogy, introduced by \citet{ullmanMindGamesGame2017} extends beyond mere perception, encompassing imagination, dreams, and memory. Each of these cognitive functions can be seen as manifestations of the brain's ability to generate, manipulate, and explore various scenarios and possibilities within its internal model. Dreams and imaginative constructs, while seemingly detached from reality, are composed of the same 'material' as our waking perceptions – they are all products of the brain's simulation capabilities \cite{pearsonHumanImaginationCognitive2019}.
The self, in this view, becomes both a creator and a perceiver of its subjective reality, a reality that, while grounded in the external world, is ultimately shaped by the mind's interpretative and predictive faculties.

In the following, an overview of the fundamental concepts used in this thesis are presented, focusing on program synthesis and its relevance to understanding human cognitive processes. Strengths and limitations of current models are discussed before outlining the approach of overcoming said limitations. FlowCoder \footnote{FlowCoder is available at \url{https://github.com/R1704/master_thesis}} is introduced as a proposed model for program synthesis. A computational model and implementational details are discussed. Two experiments are outlined and their results are analyzed. Finally, improvements and various implications of the model are highlighted.

\subsection{Background}
Traditionally, \acrfull{ai} research has been approached from two general directions. \acrfull{gofai} is based on symbolic reasoning. Symbols have no internal structure but gain significance in relation to other symbols. Models based on formal reasoning are said to be precise and tend to generalize well, yet they are slow and inflexible. Instead, deep-learning relies on distributed vector representations that have a similarity structure and facilitate analogical reasoning, which may be a core function of cognition \cite{bengio2021deep,hofstadter2013surfaces}. These models tend not to generalize well to \acrfull{ood} data and are notoriously data-hungry.
Moreover, composition, systematic generalization (\acrshort{ood}), and abstraction are often argued to be crucial aspects of human cognition \cite{cholletMeasureIntelligence2019, lecun2022path,Fodor_Pylyshyn_1988, hofstadter2013surfaces, boicho2001analogical}, which may be facilitated by a latent innate capacity for the representation and construction of part-whole hierarchies 
\cite{berwickPovertyStimulusRevisited2011,fristonWorldModelLearning2021,hintonHowRepresentPartwhole2021,martinsHowChildrenPerceive2014,raussWhatBottomUpWhat2013,schwartzBehavioralNeuralConstraints2017}.

\paragraph*{Language of Thought}\label{subsubsec:pplot}
\citet{dehaeneSymbolsMentalPrograms2022} posit that human cognition is uniquely characterized by its ability to form symbolic representations and recursive mental structures akin to a \acrfull{lot}, enabling the creation of domain-specific conceptual systems. This cognitive ability allows for the generation of new concepts through the compositional arrangement of existing elements, a process exemplified by the derivation of geometric concepts \cite{alroumiMentalCompressionSpatial2021}. Cognition simplifies complex patterns into mental representations via mental compression, where the complexity of a concept is measured by the length of its mental representation as per the \acrfull{mdl} principle. 

To illustrate, when learning to play chess, rather than remembering as many games as possible, we capture the few rules, through which we can understand and explain all instances of the game.

\begin{figure}[H]
    \centering
    \includegraphics[width=0.7\textwidth]{../img/DSL.png}
    \caption{Human cognition is underpinned by multiple mental \acrfullpl{dsl}. Each language has basic building blocks - primitives which can be programmatically composed to form more complex structures. \citet{dehaeneSymbolsMentalPrograms2022} distinguish between symmetric and asymmetric programming styles. The design principles of these mental languages are shared. They are symbolic, recursive, compositional, use formal grammar, and compress programs by adhering to the minimal description length principle. The diagram was taken from the original paper \cite{dehaeneSymbolsMentalPrograms2022}.}
    \label{fig:DSL}
\end{figure}

Current versions of the \acrshort{lot} posit that the brain implements mechanisms analogous to those found in probabilistic programming languages, enabling it to represent and infer the probabilistic structure of the world \cite{lakeBuildingMachinesThat2017,ruleChildHacker2020}. A program here can be thought of a procedure that generates more examples of the same concept. If a program would represent the concept "animal", it would generate examples such as "giraffe", "zebra", "fish", and so on. Higher-level programs could produce lower-level programs. In this paradigm, the essential aspect of compositionality gives rise to a part-whole hierarchical structure, which facilitates systematic generalization.

\paragraph*{Program Synthesis and Problem Statement}
This computational model of cognition can be formalized as \emph{program synthesis}, where the goal is to automatically construct programs that satisfy a given set of specifications.
Program synthesis involves defining a domain-specific language with a set of primitives and rules, and then searching within this language for a program that satisfies a given set of input-output relations, representing the task at hand. This process is fundamentally about mapping a defined task to an executable program within the constraints of the specified \acrshort{dsl}.

A Domain-Specific Language \( \mathcal{D} \) is defined as a set of syntactic and semantic rules that determine the structure and meaning of valid expressions in the language. Formally, a \acrshort{dsl} can be represented as:
\[ \mathcal{D} = \{ \mathcal{S}, \mathcal{O}, \mathcal{R} \} \]
where \( \mathcal{S} \) is the syntax defining the structure of valid expressions, \( \mathcal{O} \) is the set of operations (or primitives) available in the language, and \( \mathcal{R} \) are the semantic rules that assign meaning to the expressions.

Primitives in the \acrshort{dsl} are the basic operations from which programs are constructed. Each primitive \( o \in \mathcal{O} \) can be thought of as a function:
\[ o: A \rightarrow B \]
where \( A \) is the set of input types and \( B \) is the output type for the primitive.

A task \( x \in X \) in program synthesis is defined as a set of input-output pairs that specify the desired behavior of a program. Formally, a task can be represented as:
\[ x = \{ (x_{in_1}, x_{out_1}), (x_{in_2}, x_{out_2}), ..., (x_{in_n}, x_{out_n}) \} \]
where each pair \( (x_{in_i}, x_{out_i}) \) consists of an input \( x_{in_i} \) and the corresponding desired output \( x_{out_i} \).
The objective of program synthesis is to find a program \( \rho \) within the language \( \mathcal{D} \) that satisfies the task \( x \). Formally, this can be seen as a search problem:
Find \( \rho \in \mathcal{D} \) such that for every \( (x_{in_i}, x_{out_i}) \in x \), \( \rho(x_{in_i}) = x_{out_i} \).

\paragraph*{DreamCoder}\label{subsubsec:dreamcoder}
\acrfull{dc} stands out as a particularly effective model in program synthesis, creating programs from basic primitives and tasks with the goal of developing its own domain-specific language \cite{ellisDreamCoderBootstrappingInductive2021}. It employs an adapted wake-sleep algorithm, initially introduced by \citet{hinton1995wake}, to simultaneously train a generative model and a recognition network. The generative model is tasked with learning a probability distribution across programs, while the recognition network is designed to map tasks to specific programs, facilitating a neurally-guided exploration of the program space. This process leverages the recognition network to implement a parallel search strategy, blending best-first and depth-first searches to prioritize programs based on their probabilities.

The model significantly narrows the search scope by abstracting frequently used sub-routines into more readily accessible concepts, thereby enhancing scalability. This abstraction not only reduces the depth of the search tree but also limits its breadth, with the abstraction phase playing a pivotal role in refactoring subroutines in accordance with the \acrlong{mdl} principle and in the learning of the \acrlong{dsl}.

The tasks addressed by DreamCoder can either be generative, such as image creation, or conditional, like establishing input-output relationships for list sorting. Examples of tasks from various domains are depicted in \autoref{fig:conc_library}(A), while \autoref{fig:conc_library}(B) illustrates the process of learning to sort a list. The figure shows initial primitives on the left, a middle section highlighting the library of learned concepts and the established part-whole hierarchy, and on the right, the ultimate solution employing \texttt{concept15}, which itself incorporates previously abstracted concepts.

\begin{figure}[H]
    \centering
    \includegraphics[width=\textwidth]{../img/conc_library.png}
    \caption{(A) Tasks across eight distinct domains. (B) Illustration of the concept library that has been acquired. The left side displays the foundational primitives that are used to construct the concepts shown in the central area. To the right, a task is presented through input-output relationships alongside the derived solution. Below, this solution is reformulated using solely the initial primitives. Image taken with permission from the original paper \cite{ellisDreamCoderBootstrappingInductive2021}.}
    \label{fig:conc_library}
\end{figure}


\paragraph*{DeepSynth}

\citet{fijalkowScalingNeuralProgram2021} propose a framework called "distribution-based search", in which they investigate the difficult problem of searching through a \acrshort{dsl} to find programs matching a specification in a vast hypothesis space.
They introduce DeepSynth \footnote{\url{https://github.com/nathanael-fijalkow/DeepSynth}}, a general-purpose program synthesizer which constructs programs from input-output examples, and a useful framework allowing us to test different models and search methods, which I am using in this project.
The authors discuss different program finding strategies. Specifically, they find that both enumerative search (as in \acrshort{dc}) and sampling are viable strategies, where search is associated with prioritizing quantity, i.e. creating many programs quickly, whereas sampling strategies prioritize quality but may be slower, since resampling may occur. An additional benefit of sampling over search is space efficiency - already created programs don't need to be memorized.
Here, an initial \acrshort{dsl} along with suitable syntactic constraints compile into a \acrfull{cfg}, defining the possible structures of programs within its \acrshort{dsl}. A \acrshort{cfg} consists of a set of production rules that describe how to generate strings from a set of non-terminal and terminal symbols. It is "context-free" because the production rules are applied regardless of the surrounding symbols.
In DeepSynth, a prediction model is used to predict weights for a \acrfull{pcfg}, extending the \acrshort{cfg} by associating probabilities with the production rules. This allows the grammar to not only generate the syntactic structure of a program but also to represent beliefs about the relative plausibility or frequency of different structures \footnote{See appendix \autoref{app:cfg} for a formalization of \acrshortpl{cfg} and \acrshortpl{pcfg}.}. This is similar to DreamCoder's prior, consisting of a library of sub-routines combined with a weight vector. The \acrshort{pcfg} guides the search and inference process towards more likely programs. DreamCoder however, does not specifically use a \acrshort{pcfg}. Both frameworks employ a typed $\lambda$-calculus, hence there are restrictions on program arguments, etc. (syntactical constraints). DreamCoder performs type inference during program generation. To spare computational cost, DeepSynth constructs the \acrshort{cfg} beforehand which in turn increases its size.
\citet{fijalkowScalingNeuralProgram2021} compare different search strategies and show that methods that do not use a machine-learned \acrshort{pcfg} (e.g. \acrfull{dfs}) barely solve any tasks, demonstrating the necessity for better strategies.

\subsection{Limitations}
Although DreamCoder and DeepSynth prove to be successful in synthesizing programs, their methods reveal a foundational limitation: their heavy reliance on syntactical constraints.
While these constraints are undoubtedly vital for ensuring the correctness of generated programs, they do not necessarily guarantee a deep understanding or utilization of semantic relationships within the code. Additionally, \citet{kimCompoundProbabilisticContextFree2019} explain that associating only a scalar per rule misses a lot of information. A distributed representation of the \acrshort{dsl} would therefore be beneficial. We could imagine a program space in which certain symmetries could be leveraged. One could argue e.g. that "\(+\)" is to "\(-\)" as "\(\div\)" is to "\(\times\)". These semantic relationships may be missed in the previously discussed models.

\subsection{Approach}
\paragraph*{Transformers and Self-Attention}
The Transformer architecture, originally introduced in 2017 by \citet{vaswaniAttentionAllYou2017}, has proved to be widely successful in a wide range of applications \cite{wolfTransformersStateoftheArtNatural2020,khanTransformersVisionSurvey2022}. Transformers use self-attention, a mechanism that enables dynamic selection and focus on specific parts of the input, as opposed to treating all parts equally. It effectively allows the network to "attend" to, or give more weight to, certain inputs over others during the processing stage. The self-attention mechanism allows for an understanding of not just the structural arrangement of elements in a sequence (syntax) but also their deeper, contextual relationships (semantics) \cite{wolfram2023chatgpt}. In this thesis I will use this model for a rich representation of programs. However, training the Transformer is difficult from only a few examples. Therefore, I will combine the approach with an amortized sampler, explained in the following.

\paragraph*{GFlowNet}
\acrfullpl{gfn}, introduced by \citet{bengioFlowNetworkBased2021}, are a class of generative models designed to learn to construct compositional objects from a target distribution over complex high-dimensional spaces, particularly where explicit density estimation is challenging and diverse candidates are encouraged. \acrshortpl{gfn} learn a stochastic policy for generating sequences of actions that lead to the construction of a sample. The model generates sequences of actions that build a sample, with the generation frequency of each sample being proportional to an associated reward function.
In other words, \acrshortpl{gfn} are applicable in problems where complex structures are composed from simple building blocks and have been used in molecular composition from atoms \cite{bengioFlowNetworkBased2021}, in grammar induction \cite{Hu_Malkin_Jain_Everett_Graikos_Bengio_2023}, and in Bayesian structure learning \cite{deleuBayesianStructureLearning2022}. The learnt policy becomes an amortized sampler. This means that the extensive training invested in the model results in a system capable of efficiently generating new samples without the need for additional, extensive computation for each new instance. Moreover, the model can be used for offline training, i.e. from data that is not from the observed distribution. This aspect is crucial for \acrshort{ood} generalization and may be the remedy for data-hungry Transformers.

\subsection{Research Question, Aim, Motivation}
In recent advancements, \acrfull{sota} models like \acrlong{dc} have demonstrated proficiency in program synthesis. However, they often lack a semantically rich state representation and heavily rely on search algorithms for constructing programs. This thesis aims to investigate a novel approach by combining the strengths of two distinct architectures: the Transformer and \acrshort{gfn}. The Transformer architecture is known for its ability to learn rich state spaces, albeit with a significant data requirement and limited generalization to \acrlong{ood} tasks. On the other hand, \acrshort{gfn}, with its capability for amortized sampling, presents a promising solution to overcome these challenges. The central hypothesis of this thesis is that the integration of these two architectures could yield a powerful program synthesizer. This synthesizer would be capable of solving tasks with minimal examples, specifically in the list-editing domain. 
Furthermore, theoretical and computational challenges are identified and addressed within the realm of neural program synthesis.

This research explores the potential alignment of the proposed model with the \acrlong{lot} hypothesis, suggesting a programming language-like mental representation underpinning human thought. 
Thus, this research not only aims to address a practical gap in program synthesis but to explore the role of program synthesis in a model of cognition, thereby contributing to the philosophical and psychological understanding of thought, and intelligence. 

\subsection{Scope and Limitations}
The concept of abstraction in program synthesis is necessary for the model to learn its own \acrshort{dsl}. Abstraction effectively narrows the depth of the search tree through program refactoring and identifies common patterns, thereby aiding in generalization. Additionally, abstraction is essential in optimizing for the \acrlong{mdl}, which is a useful inductive bias humans seem to employ \cite{sable-meyerLanguageThoughtMental2022}.
However, in this research, abstraction was not implemented. This decision was primarily guided by time constraints. As a result, I focus on modeling a program synthesizer that solves tasks and on testing its abilities, rather than additionally learning the \acrshort{dsl}. Consequently, it is anticipated that the model will not optimize for parsimonious programs.


% \subsection{\red{Main Contributions}}
% FlowCoder \footnote{Code available at \url{https://github.com/R1704/master_thesis}}.
% % Highlight the significance of your research within the field of AI. Explain how your work contributes to advancing knowledge or addressing the identified research gap. Mention the potential impact of your findings or proposed solutions.
% % The main contributions of this thesis are: 
% % \begin{itemize}
% %     \item I argue that the same multi-scale hierarchies that govern biological organization can be applied to the conceptual realm
% %     \item I develop a method of bayesian program synthesis, using GFlowNet
% %     \item I present the philosophical ramifications
% % \end{itemize}

% % \begin{itemize}
% %     \item novel program synthesizer FlowCoder
% %     \item philosophical ramifications as a model of cognition
% %     \item results 
% %     \item conclusion.
% % \end{itemize}





% I present a novel method for program synthesis and show that systematic generalization is achievable without explicit world models, i.e. without parameterizing a \acrshort{dsl} or \acrshort{pcfg}. Rather, implicit world models in the form of Transformers may be sufficient. I discuss the consequences of the two variants and propose extensions to my model. 

\subsection{Problem Statement}



\section{Computation and the Computational Mind}
\section{Language of Thought}
\section{Grammar}
- GOFAI
- Connectionism
- Hybrid
\section{Syntax vs Semantics}



Let's assume the premise that the mind is computational, in one form or another. 
and we are building a world model which can be thought of as probabilistic programs. 

\section{Program Synthesis}

Probabilistic programs, in essence, represent a form of causal reasoning. By representing beliefs as probability distributions and reasoning patterns as programmatic structures, they offer a nuanced way of modeling complex real-world systems. Given the inherently uncertain nature of our environment and the myriad possible interpretations of sensory data, a probabilistic approach is naturally aligned with the cognitive demands of building accurate and adaptable world models.

**3. Program Synthesis: A Key to Unlocking Cognition**

Program synthesis refers to the automatic generation of programs from a higher-level specification. In the context of cognition, it implies a process where an agent—human or machine—generates a novel programmatic structure to represent or reason about its environment. Understanding this synthesis process becomes crucial for several reasons:

* *Generativity and Flexibility:* Human cognition is not merely reactive. It proactively creates, hypothesizes, and experiments. By studying how programs are synthesized, we might glean insights into the generative aspects of thought.

* *Abstraction:* At the heart of program synthesis is the ability to abstract away from particulars and generate general rules or patterns. This mirrors cognitive abilities like generalization and analogy-making.

* *Efficiency:* Just as efficient algorithms are prized in computing, efficient cognitive strategies are essential for survival. Program synthesis might hold the key to understanding how humans prune vast possibility spaces to arrive at functional solutions quickly.

Certainly! Here's the problem statement with LaTeX formatting for mathematical expressions:

---

**Problem Statement: Navigating the Vast Program Search Space Within Context-Free Grammars**

---

**1. Background and Introduction**

The field of programming has always been underpinned by the intricacies of formal grammars. Context-Free Grammars (CFGs), a subset of formal grammar, are essential in defining the syntactical structures of many programming languages. However, given the generative nature of CFGs, the potential program space defined by even a modestly complex grammar can be immensely vast. Searching for a specific program within this space, or ensuring that a particular space is sufficiently explored, poses significant computational challenges.

**2. Problem Definition**

**2.1 CFG and Program Space**

Let \( G = (N, \Sigma, P, S) \) be a Context-Free Grammar, where:
\begin{itemize}
    \item \( N \) is a finite set of non-terminal symbols.
    \item \( \Sigma \) is a finite set of terminal symbols with \newline \( N \cap \Sigma = \emptyset \)
    \item \( P \) is a finite set of production rules, where each rule is of the form \( N \rightarrow (N \cup \Sigma)^* \)
    \item \( S \) is the start symbol, with \( S \in N \)
\end{itemize}

Given such a CFG, the derived program space \( \Pi(G) \) is the set of all possible strings (or sequences of symbols) derivable from \( S \).

**2.2 Problem Statement**

Given a Context-Free Grammar \( G \) and a defined objective function \( f \) that maps any program \( p \in \Pi(G) \) to a real value representing its desirability or fitness:

*Find \( p^* \) such that:*
\[ p^* = \arg\max_{p \in \Pi(G)} f(p) \]

In other words, the problem is to locate a program \( p^* \) within the vast program space \( \Pi(G) \) defined by \( G \) that maximizes (or, alternatively, minimizes) the objective function \( f \).

**3. Challenges and Complications**

**3.1 Size of the Search Space**

The generative capacity of CFGs means that even grammars of moderate complexity can define immensely vast program spaces. The sheer size of these spaces poses computational and search challenges.

**3.2 Non-Linearity and Discontinuities**

The mapping between programs and their fitness as defined by \( f \) might be non-linear with multiple local maxima, making search strategies based on gradient ascent or other linear heuristics suboptimal.

**3.3 Generalization vs Specialization**

While CFGs provide a generalized representation of possible programs, the objective function might lead to highly specialized solutions. Balancing between the two is non-trivial.

**3.4 Syntactic vs Semantic Validity**

A CFG ensures syntactic validity but does not guarantee semantic correctness. Ensuring that a program derived from a CFG is semantically meaningful or error-free in a given context is an additional layer of complexity.

**4. Significance**

Solving or even approximating solutions for this problem has far-reaching consequences. It touches on fields from program synthesis, where specific algorithms or code snippets are automatically generated to meet specific requirements, to genetic programming, where evolutionary methods are employed to 'evolve' optimal or near-optimal solutions.

Additionally, insights from this exploration can impact compiler design, optimization strategies in high-performance computing, and even areas like natural language processing, where CFGs have historically played a foundational role.

**5. Conclusion**

The exploration of the vast program space generated by CFGs and efficiently searching within it for optimal programs represents a formidable challenge. It is a nexus of computational theory, practical programming, and numerous applied domains. Addressing it promises not just solutions to specific computational problems but also deeper insights into the nature of computation, representation, and optimization.

**4. Techniques in Program Synthesis**

Several techniques have emerged as pivotal in the domain of program synthesis:

* *Deductive Synthesis:* Rooted in formal logic, this method transforms specifications into programs. The use of logic mirrors certain cognitive tasks, especially those demanding strict reasoning.

* *Stochastic Search:* By randomly exploring the space of possible programs, these methods mirror heuristic-based cognitive processes. Genetic algorithms, for instance, mimic evolutionary processes to evolve optimal or near-optimal solutions.

* *Neural Program Synthesis:* Neural networks, especially recurrent ones, have shown promise in generating programmatic structures. The parallels between neural networks and neural structures in the brain offer tantalizing possibilities for cognitive science.

* *Example-Based Synthesis:* Drawing inspiration from how humans often learn—from examples—these methods generate programs by generalizing from provided instances. This mirrors pedagogical processes and experiential learning.



\section{DreamCoder}
A Bayesian View
--> What exactly happens?
\section{DeepSynth}
--> Introducing different search strategies.
\section{GFlowNet}
A generative policy




(Probabilistic) Context free grammar. 
\subsection{Previous work on program synthesis.}
Check out microsoft paper.
DreamCoder
DeepSynth
\section{Search space}
\section{Program space}
- How can we calculate the size of that space?
- Hierarchical structure 
- Abstract Syntax Trees

\section{GFlowNet}
-







In a probabilistic programming representation of world models and thoughts, as a type of Language of Thought, which aligns with constructivism we can conceive of the problem statement as 
We want to construct objects.

In a language of thought, regardless of most details, we tend to think in a paradigm in which thoughts are compositional. In one way or another, thoughts are hierarchical.
Both in ontologies, i.e. the way concepts are structured, (animal -> bird -> fink) but also in the sequential nature of thought construction (as in natural language, reasoning tasks, etc.). We are creating parse trees, or abstract syntax trees. 

In a probabilistic programming paradigm, we view this as compositional functions, with the many properties of functional programming analogous to currying, etc. 

[Thoughts as trajectories]

We can formalise this as a hierarchical latent structure Z.

Framing the problem.

Combinatorial search problem. 

- If we are using encoder + decoder, are we violating the Markovian Flow assumption?
    Look at the smiley example. If the NN would get [[left brow], [left brow, right brow], [left brow, right brow, smile]] as input, i.e. the sequence of the states, it would violate the assumption. but it only gets the current state, e.g. [left brow, right brow], and from that it has to infer the next step. 
    When using a decoder, we would indeed give it the whole trajectory of states, so it would violate the assumption.
    But even now i am encoding the sequence and giving the whole trajectory as input. and since the CFG is essentially a tree, there is only one parent for each state. 

It would probably be faster to do it bottom up like in HEAP search from Nathanael and from GFN-EM, then we could also use sub-trajectory balance, i.e. calculate intermediate Rewards. 

Another thing we could do is predict a bunch of terminals at once, and then combine in each step. 

Other ideas: wave function collapse

\section{assumptions}
\section{limitations}
\section{biological plausibility}

\subsection{Purpose of the Study}
\subsection{Relevance of the Study}

\section{Theoretical Framework}
\subsection{Definitions}
\begin{itemize}
        \item What is a cognitive agent?
        \item What is the mind?
        \item What is the self?
        \item What is the "I"?
        \item What is a thought? (GFN trajectory, GPT trajectory, trajectory of some sort)
\end{itemize}
\subsection{The Need for Discernment in Cognitive Agents}
The concept of discernment in cognitive agents is fundamental to understanding cognition. In essence, discernment is the ability of an agent to differentiate and distinguish between various elements within its environment. Without discernment, an agent would be unable to make sense of its world or make decisions to optimize its performance within that world. Discernment, therefore, lies at the foundation of cognition, allowing the agent to separate itself from its surroundings and engage meaningfully with the world.
\subsection{The Concept of Identity and Attribution of Cause and Effect, in relation to the self}
\subsubsection{Karl Friston's perspective}
\subsubsection{Michael Levin's perspective}
\subsection{The Necessity of World Modeling}
\paragraph{Representationalism}
Representationalism suggests that cognitive agents create internal representations or models of the world to understand and interact effectively with their environment. These models, while never fully capturing the complexity of the external world, provide a means for the agent to anticipate and respond to future scenarios.
\paragraph{Contrast to Statistical Pattern Recognition}
Unlike statistical pattern recognition, which involves extracting patterns and regularities from data without necessarily understanding the underlying structures, model-based reasoning involves creating and manipulating symbolic models of the world. The advantage of this approach is that it allows for a more comprehensive understanding of the world and more flexible and adaptive behavior.
\subsection{Alternative Approaches to Cognition}
\subsubsection{Enactivism}
Enactivism posits that cognition arises from the dynamic interaction between an organism and its environment. Rather than relying on internal representations, cognition is seen as a process of embodied action, where knowledge is gained through physical interaction with the world.
\subsubsection{Embodiment}
The theory of embodiment emphasizes the role of the body in shaping the mind. According to this view, many features of cognition are shaped by aspects of the body beyond the brain, including sensory and motor systems. This perspective encourages a holistic view of cognition, bridging the mind-body divide.
\subsection{Analogy as the Core of Cognition: Douglas Hofstadter's Ideas}
\subsection{Virtualism and the "Game Engine in the Head": Perspectives of Joscha Bach and Josh Tenenbaum}

\section{The Art and Science of Model-making}
\subsection{Symbolic vs. Sub-symbolic Approaches}
\subsection{Analytic vs. Geometric Solutions}
\subsection{System 1 \& 2: Daniel Kahneman's theories and relation to Models}
\subsection{Hemispheric Lateralization: Iain McGilchrist's Theory}

\section{Spaces in Cognition}
\subsection{Latent Spaces}
\subsection{Cognitive and Conceptual Spaces}
\subsubsection{Peter Gärdenfors's Perspective}
\subsection{The Geometry of Cognitive Spaces}
\subsubsection{Hyperbolic Spaces}

\section{Computationalism and Computational Theories of the Universe}
\subsection{Computational Mind}
\subsection{Theories of Stephen Wolfram and David Deutsch}
\subsection{Discussion on the Nature of Reality: Continuous vs. Discrete}

\section{The Language of Thought}
\subsection{Gödel's Incompleteness Theorem and Turing's Extensions}
\subsection{Chomsky's Hierarchy}
\subsection{Computational Irreducibility}
\subsection{Constructivism}
\subsection{Syntax vs. Semantics in Language}
\subsection{Semiotics}
\subsection{The Notion of a Semantic Grammar}

\section{The Nature of Meaning}
\subsection{Meaning in the context of Modeling Approaches}
\subsection{The Grounding Problem: Harnard's perspective}
\subsection{Concept of Meaningfulness}

\section{The Formation and Representation of Concepts}
\subsection{Approaches to Representation}
\subsection{The Theory of Memes: Richard Dawkins's perspective}
\subsection{Concepts as a Self-Organizing System}
\subsection{Connection to Cellular Automata}

\section{Representation of Concepts as Probabilistic Programs}
\subsection{Overview of Probabilistic Programs}
\subsection{Role of Program Synthesis}

\section{Conclusion}
\subsection{Summary of Findings}
\subsection{Implications for the Field}
\subsection{Recommendations for Future Research}




\section{Introduction}
Imagine your brain as an interactive game engine. Just as a game engine generates dynamic virtual environments, complete with rules and physics that players interact with, the brain constructs a model of the real world. This model includes rules (physical laws, social norms), entities (objects, people), and interactions (how things work and relate to each other). We learn to navigate and predict our environment, constantly updating our internal model based on new experiences and information.
This analogy, introduced by \citet{ullmanMindGamesGame2017} extends beyond mere perception, encompassing imagination, dreams, and memory. Each of these cognitive functions can be seen as manifestations of the brain's ability to generate, manipulate, and explore various scenarios and possibilities within its internal model. Dreams and imaginative constructs, while seemingly detached from reality, are composed of the same 'material' as our waking perceptions – they are all products of the brain's simulation capabilities \cite{pearsonHumanImaginationCognitive2019}.
The self, in this view, becomes both a creator and a perceiver of its subjective reality, a reality that, while grounded in the external world, is ultimately shaped by the mind's interpretative and predictive faculties.

In the following, an overview of the fundamental concepts used in this thesis are presented, focusing on program synthesis and its relevance to understanding human cognitive processes. Strengths and limitations of current models are discussed before outlining the approach of overcoming said limitations. FlowCoder \footnote{FlowCoder is available at \url{https://github.com/R1704/master_thesis}} is introduced as a proposed model for program synthesis. A computational model and implementational details are discussed. Two experiments are outlined and their results are analyzed. Finally, improvements and various implications of the model are highlighted.

\subsection{Background}
Traditionally, \acrfull{ai} research has been approached from two general directions. \acrfull{gofai} is based on symbolic reasoning. Symbols have no internal structure but gain significance in relation to other symbols. Models based on formal reasoning are said to be precise and tend to generalize well, yet they are slow and inflexible. Instead, deep-learning relies on distributed vector representations that have a similarity structure and facilitate analogical reasoning, which may be a core function of cognition \cite{bengio2021deep,hofstadter2013surfaces}. These models tend not to generalize well to \acrfull{ood} data and are notoriously data-hungry.
Moreover, composition, systematic generalization (\acrshort{ood}), and abstraction are often argued to be crucial aspects of human cognition \cite{cholletMeasureIntelligence2019, lecun2022path,Fodor_Pylyshyn_1988, hofstadter2013surfaces, boicho2001analogical}, which may be facilitated by a latent innate capacity for the representation and construction of part-whole hierarchies 
\cite{berwickPovertyStimulusRevisited2011,fristonWorldModelLearning2021,hintonHowRepresentPartwhole2021,martinsHowChildrenPerceive2014,raussWhatBottomUpWhat2013,schwartzBehavioralNeuralConstraints2017}.

\paragraph*{Language of Thought}\label{subsubsec:pplot}
\citet{dehaeneSymbolsMentalPrograms2022} posit that human cognition is uniquely characterized by its ability to form symbolic representations and recursive mental structures akin to a \acrfull{lot}, enabling the creation of domain-specific conceptual systems. This cognitive ability allows for the generation of new concepts through the compositional arrangement of existing elements, a process exemplified by the derivation of geometric concepts \cite{alroumiMentalCompressionSpatial2021}. Cognition simplifies complex patterns into mental representations via mental compression, where the complexity of a concept is measured by the length of its mental representation as per the \acrfull{mdl} principle. 

To illustrate, when learning to play chess, rather than remembering as many games as possible, we capture the few rules, through which we can understand and explain all instances of the game.

\begin{figure}[H]
    \centering
    \includegraphics[width=0.7\textwidth]{../img/DSL.png}
    \caption{Human cognition is underpinned by multiple mental \acrfullpl{dsl}. Each language has basic building blocks - primitives which can be programmatically composed to form more complex structures. \citet{dehaeneSymbolsMentalPrograms2022} distinguish between symmetric and asymmetric programming styles. The design principles of these mental languages are shared. They are symbolic, recursive, compositional, use formal grammar, and compress programs by adhering to the minimal description length principle. The diagram was taken from the original paper \cite{dehaeneSymbolsMentalPrograms2022}.}
    \label{fig:DSL}
\end{figure}

Current versions of the \acrshort{lot} posit that the brain implements mechanisms analogous to those found in probabilistic programming languages, enabling it to represent and infer the probabilistic structure of the world \cite{lakeBuildingMachinesThat2017,ruleChildHacker2020}. A program here can be thought of a procedure that generates more examples of the same concept. If a program would represent the concept "animal", it would generate examples such as "giraffe", "zebra", "fish", and so on. Higher-level programs could produce lower-level programs. In this paradigm, the essential aspect of compositionality gives rise to a part-whole hierarchical structure, which facilitates systematic generalization.

\paragraph*{Program Synthesis and Problem Statement}
This computational model of cognition can be formalized as \emph{program synthesis}, where the goal is to automatically construct programs that satisfy a given set of specifications.
Program synthesis involves defining a domain-specific language with a set of primitives and rules, and then searching within this language for a program that satisfies a given set of input-output relations, representing the task at hand. This process is fundamentally about mapping a defined task to an executable program within the constraints of the specified \acrshort{dsl}.

A Domain-Specific Language \( \mathcal{D} \) is defined as a set of syntactic and semantic rules that determine the structure and meaning of valid expressions in the language. Formally, a \acrshort{dsl} can be represented as:
\[ \mathcal{D} = \{ \mathcal{S}, \mathcal{O}, \mathcal{R} \} \]
where \( \mathcal{S} \) is the syntax defining the structure of valid expressions, \( \mathcal{O} \) is the set of operations (or primitives) available in the language, and \( \mathcal{R} \) are the semantic rules that assign meaning to the expressions.

Primitives in the \acrshort{dsl} are the basic operations from which programs are constructed. Each primitive \( o \in \mathcal{O} \) can be thought of as a function:
\[ o: A \rightarrow B \]
where \( A \) is the set of input types and \( B \) is the output type for the primitive.

A task \( x \in X \) in program synthesis is defined as a set of input-output pairs that specify the desired behavior of a program. Formally, a task can be represented as:
\[ x = \{ (x_{in_1}, x_{out_1}), (x_{in_2}, x_{out_2}), ..., (x_{in_n}, x_{out_n}) \} \]
where each pair \( (x_{in_i}, x_{out_i}) \) consists of an input \( x_{in_i} \) and the corresponding desired output \( x_{out_i} \).
The objective of program synthesis is to find a program \( \rho \) within the language \( \mathcal{D} \) that satisfies the task \( x \). Formally, this can be seen as a search problem:
Find \( \rho \in \mathcal{D} \) such that for every \( (x_{in_i}, x_{out_i}) \in x \), \( \rho(x_{in_i}) = x_{out_i} \).

\paragraph*{DreamCoder}\label{subsubsec:dreamcoder}
\acrfull{dc} stands out as a particularly effective model in program synthesis, creating programs from basic primitives and tasks with the goal of developing its own domain-specific language \cite{ellisDreamCoderBootstrappingInductive2021}. It employs an adapted wake-sleep algorithm, initially introduced by \citet{hinton1995wake}, to simultaneously train a generative model and a recognition network. The generative model is tasked with learning a probability distribution across programs, while the recognition network is designed to map tasks to specific programs, facilitating a neurally-guided exploration of the program space. This process leverages the recognition network to implement a parallel search strategy, blending best-first and depth-first searches to prioritize programs based on their probabilities.

The model significantly narrows the search scope by abstracting frequently used sub-routines into more readily accessible concepts, thereby enhancing scalability. This abstraction not only reduces the depth of the search tree but also limits its breadth, with the abstraction phase playing a pivotal role in refactoring subroutines in accordance with the \acrlong{mdl} principle and in the learning of the \acrlong{dsl}.

The tasks addressed by DreamCoder can either be generative, such as image creation, or conditional, like establishing input-output relationships for list sorting. Examples of tasks from various domains are depicted in \autoref{fig:conc_library}(A), while \autoref{fig:conc_library}(B) illustrates the process of learning to sort a list. The figure shows initial primitives on the left, a middle section highlighting the library of learned concepts and the established part-whole hierarchy, and on the right, the ultimate solution employing \texttt{concept15}, which itself incorporates previously abstracted concepts.

\begin{figure}[H]
    \centering
    \includegraphics[width=\textwidth]{../img/conc_library.png}
    \caption{(A) Tasks across eight distinct domains. (B) Illustration of the concept library that has been acquired. The left side displays the foundational primitives that are used to construct the concepts shown in the central area. To the right, a task is presented through input-output relationships alongside the derived solution. Below, this solution is reformulated using solely the initial primitives. Image taken with permission from the original paper \cite{ellisDreamCoderBootstrappingInductive2021}.}
    \label{fig:conc_library}
\end{figure}


\paragraph*{DeepSynth}

\citet{fijalkowScalingNeuralProgram2021} propose a framework called "distribution-based search", in which they investigate the difficult problem of searching through a \acrshort{dsl} to find programs matching a specification in a vast hypothesis space.
They introduce DeepSynth \footnote{\url{https://github.com/nathanael-fijalkow/DeepSynth}}, a general-purpose program synthesizer which constructs programs from input-output examples, and a useful framework allowing us to test different models and search methods, which I am using in this project.
The authors discuss different program finding strategies. Specifically, they find that both enumerative search (as in \acrshort{dc}) and sampling are viable strategies, where search is associated with prioritizing quantity, i.e. creating many programs quickly, whereas sampling strategies prioritize quality but may be slower, since resampling may occur. An additional benefit of sampling over search is space efficiency - already created programs don't need to be memorized.
Here, an initial \acrshort{dsl} along with suitable syntactic constraints compile into a \acrfull{cfg}, defining the possible structures of programs within its \acrshort{dsl}. A \acrshort{cfg} consists of a set of production rules that describe how to generate strings from a set of non-terminal and terminal symbols. It is "context-free" because the production rules are applied regardless of the surrounding symbols.
In DeepSynth, a prediction model is used to predict weights for a \acrfull{pcfg}, extending the \acrshort{cfg} by associating probabilities with the production rules. This allows the grammar to not only generate the syntactic structure of a program but also to represent beliefs about the relative plausibility or frequency of different structures \footnote{See appendix \autoref{app:cfg} for a formalization of \acrshortpl{cfg} and \acrshortpl{pcfg}.}. This is similar to DreamCoder's prior, consisting of a library of sub-routines combined with a weight vector. The \acrshort{pcfg} guides the search and inference process towards more likely programs. DreamCoder however, does not specifically use a \acrshort{pcfg}. Both frameworks employ a typed $\lambda$-calculus, hence there are restrictions on program arguments, etc. (syntactical constraints). DreamCoder performs type inference during program generation. To spare computational cost, DeepSynth constructs the \acrshort{cfg} beforehand which in turn increases its size.
\citet{fijalkowScalingNeuralProgram2021} compare different search strategies and show that methods that do not use a machine-learned \acrshort{pcfg} (e.g. \acrfull{dfs}) barely solve any tasks, demonstrating the necessity for better strategies.

\subsection{Limitations}
Although DreamCoder and DeepSynth prove to be successful in synthesizing programs, their methods reveal a foundational limitation: their heavy reliance on syntactical constraints.
While these constraints are undoubtedly vital for ensuring the correctness of generated programs, they do not necessarily guarantee a deep understanding or utilization of semantic relationships within the code. Additionally, \citet{kimCompoundProbabilisticContextFree2019} explain that associating only a scalar per rule misses a lot of information. A distributed representation of the \acrshort{dsl} would therefore be beneficial. We could imagine a program space in which certain symmetries could be leveraged. One could argue e.g. that "\(+\)" is to "\(-\)" as "\(\div\)" is to "\(\times\)". These semantic relationships may be missed in the previously discussed models.

\subsection{Approach}
\paragraph*{Transformers and Self-Attention}
The Transformer architecture, originally introduced in 2017 by \citet{vaswaniAttentionAllYou2017}, has proved to be widely successful in a wide range of applications \cite{wolfTransformersStateoftheArtNatural2020,khanTransformersVisionSurvey2022}. Transformers use self-attention, a mechanism that enables dynamic selection and focus on specific parts of the input, as opposed to treating all parts equally. It effectively allows the network to "attend" to, or give more weight to, certain inputs over others during the processing stage. The self-attention mechanism allows for an understanding of not just the structural arrangement of elements in a sequence (syntax) but also their deeper, contextual relationships (semantics) \cite{wolfram2023chatgpt}. In this thesis I will use this model for a rich representation of programs. However, training the Transformer is difficult from only a few examples. Therefore, I will combine the approach with an amortized sampler, explained in the following.

\paragraph*{GFlowNet}
\acrfullpl{gfn}, introduced by \citet{bengioFlowNetworkBased2021}, are a class of generative models designed to learn to construct compositional objects from a target distribution over complex high-dimensional spaces, particularly where explicit density estimation is challenging and diverse candidates are encouraged. \acrshortpl{gfn} learn a stochastic policy for generating sequences of actions that lead to the construction of a sample. The model generates sequences of actions that build a sample, with the generation frequency of each sample being proportional to an associated reward function.
In other words, \acrshortpl{gfn} are applicable in problems where complex structures are composed from simple building blocks and have been used in molecular composition from atoms \cite{bengioFlowNetworkBased2021}, in grammar induction \cite{Hu_Malkin_Jain_Everett_Graikos_Bengio_2023}, and in Bayesian structure learning \cite{deleuBayesianStructureLearning2022}. The learnt policy becomes an amortized sampler. This means that the extensive training invested in the model results in a system capable of efficiently generating new samples without the need for additional, extensive computation for each new instance. Moreover, the model can be used for offline training, i.e. from data that is not from the observed distribution. This aspect is crucial for \acrshort{ood} generalization and may be the remedy for data-hungry Transformers.

\subsection{Research Question, Aim, Motivation}
In recent advancements, \acrfull{sota} models like \acrlong{dc} have demonstrated proficiency in program synthesis. However, they often lack a semantically rich state representation and heavily rely on search algorithms for constructing programs. This thesis aims to investigate a novel approach by combining the strengths of two distinct architectures: the Transformer and \acrshort{gfn}. The Transformer architecture is known for its ability to learn rich state spaces, albeit with a significant data requirement and limited generalization to \acrlong{ood} tasks. On the other hand, \acrshort{gfn}, with its capability for amortized sampling, presents a promising solution to overcome these challenges. The central hypothesis of this thesis is that the integration of these two architectures could yield a powerful program synthesizer. This synthesizer would be capable of solving tasks with minimal examples, specifically in the list-editing domain. 
Furthermore, theoretical and computational challenges are identified and addressed within the realm of neural program synthesis.

This research explores the potential alignment of the proposed model with the \acrlong{lot} hypothesis, suggesting a programming language-like mental representation underpinning human thought. 
Thus, this research not only aims to address a practical gap in program synthesis but to explore the role of program synthesis in a model of cognition, thereby contributing to the philosophical and psychological understanding of thought, and intelligence. 

\subsection{Scope and Limitations}
The concept of abstraction in program synthesis is necessary for the model to learn its own \acrshort{dsl}. Abstraction effectively narrows the depth of the search tree through program refactoring and identifies common patterns, thereby aiding in generalization. Additionally, abstraction is essential in optimizing for the \acrlong{mdl}, which is a useful inductive bias humans seem to employ \cite{sable-meyerLanguageThoughtMental2022}.
However, in this research, abstraction was not implemented. This decision was primarily guided by time constraints. As a result, I focus on modeling a program synthesizer that solves tasks and on testing its abilities, rather than additionally learning the \acrshort{dsl}. Consequently, it is anticipated that the model will not optimize for parsimonious programs.


% \subsection{\red{Main Contributions}}
% FlowCoder \footnote{Code available at \url{https://github.com/R1704/master_thesis}}.
% % Highlight the significance of your research within the field of AI. Explain how your work contributes to advancing knowledge or addressing the identified research gap. Mention the potential impact of your findings or proposed solutions.
% % The main contributions of this thesis are: 
% % \begin{itemize}
% %     \item I argue that the same multi-scale hierarchies that govern biological organization can be applied to the conceptual realm
% %     \item I develop a method of bayesian program synthesis, using GFlowNet
% %     \item I present the philosophical ramifications
% % \end{itemize}

% % \begin{itemize}
% %     \item novel program synthesizer FlowCoder
% %     \item philosophical ramifications as a model of cognition
% %     \item results 
% %     \item conclusion.
% % \end{itemize}





% I present a novel method for program synthesis and show that systematic generalization is achievable without explicit world models, i.e. without parameterizing a \acrshort{dsl} or \acrshort{pcfg}. Rather, implicit world models in the form of Transformers may be sufficient. I discuss the consequences of the two variants and propose extensions to my model. 



Current AI is focused on deep learning.
difference in approaches:
- statistical pattern recognition
- models of the world

- we need very little data, 
- solve et coagula, we have concepts which we can use to dream, imagine, etc. 

- who am i? 


\subsection{Background}\label{sec:background}
- Problem statement
- program synthesis, previously \dots
- DreamCoder, DeepCoder, DeepSynth, etc. 
CFG, PCFG, etc. 

- They are predicting weights for a PCFG.



\subsubsection{Research Question}

\subsubsection{Motivation}
- Modeling human thought
\begin{description}
        \item[Test]
\end{description}
- problem statement 


\subsubsection{Impact and Importance}
- Understanding/ modeling cognition. 
- Better AI models, OOD generalization, reasoning, ...

\subsubsection{Methods}
- GFlowNet: definition, flow matching constraint, 

\subsubsection{Design}

\subsubsection{Analysis}


\section{Scientific and Societal Relevance}

\section{Conclusion}

\section{Discussion}

\appendix

\clearpage
\bibliographystyle{unsrt}
\bibliography{ref.bib}

\end{document}
