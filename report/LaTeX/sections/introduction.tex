% \chapter{Introduction}
\section{\orange{Introduction}}

In this thesis I want to investigate and discuss a certain model of cognition.
- We look at the world and learn to distinguish things. Discernment is the fundamental operator of cognition.
we create a model of the world and of ourself (res cogitans, res extensa).
- this model is like a game engine in the head, (simulation simulacrum). we have assets, etc. and using these concepts we can construct ideas, dreams, memories, reality, etc. its all made from the same stuff.



- I think that we receive sensory information and compute statistical regularities over those sensory inputs. 
- we construct a model of internal information and of external information and separate the world from the self, with ever more fine detail. - we create a generative model (reflexive) and inference model (reasoning)
- we are able to extract concepts into variables (symbols) and compute using these symbols.
- i think that the conceptual framework has a similar architecture as any self-organizing system (holarchic)
- dehaene et al show some ideas of probabilistic programming etc. 






% The pervasiveness of hierarchical and self-similar structures across various domains of natural and conceptual phenomena presents a compelling landscape for interdisciplinary inquiry. From the intricate organization of biological systems to the complex architectures of cognition and cultural dynamics, this recurrent pattern suggests an underlying principle of organization that warrants a comprehensive examination. This paper delineates the parallels between these multi-scale structures, extending the discussion into the cognitive sciences, artificial intelligence, and memetics, and situating them within the meta-theoretical framework of Wilber's Integral Theory.


% Natural systems exhibit a remarkable propensity for hierarchical organization: atoms form molecules; molecules form cells; cells form tissues; tissues form organs; and organs, in turn, form organisms. This nested structure, suggestive of a fractal-like self-similarity, is not confined to physical biology but extends into the domains of language, thought, and sociocultural constructs. Herein, we explore the implications of this hierarchical paradigm as it applies to mental models, cognitive processes, and cultural transmission, drawing correlations with existing theories and literature in these fields.


% Cognitive science posits that human thought manifests in layers of increasing complexity, from fleeting thoughts to embedded ideologies. This progression echoes biological hierarchies, whereby thoughts are the elemental units, progressively organizing into complex belief systems and comprehensive worldviews. Under the lens of the Free Energy Principle, the brain's attempt to minimize surprise through a predictive hierarchical model aligns with this multilayered structure, suggesting a cognitive economy geared towards self-organization and efficiency.


% In this thesis I am researching our ability to create a representation of the world, which leads to the ability of having a representation of ourselves and gives rise to an "I". I show how the same underlying principles describe physical organisms across multiple scales and extend that to the conceptual realm.
% A few approaches are discussed and eventually a promising Bayesian approach is experimented with and analysed. 





% \info[inline]{structure to keep in mind:}
% 3 (or more, see van Gerven) Levels of Marr
% \begin{description}
%     \item[Computational] (Building a compositional hierarchical structure, FEP, LOT, Symbolic (discrete) vs Connectionist (continuous), PPL)
%     \item[Algorithmic] (Variational Inference, GFlowNets, etc.)
%     \item[Implementational] (PyTorch, GPUs, CPUs, etc.)
% \end{description}

% \begin{itemize}
%     \item Continuously asking deeper questions as we encounter them. (what is a thought, understanding, etc.)
%     \item draw out the shape of the thesis. kind of a v shape?
%     \item golden circle (why, what, how)
%     \item But \& therefore rule of matt stone and trey parker 
% \end{itemize}

% \begin{itemize}
%     \item First we present the ubiquitous principle of nested multi-scale systems.
%     \item We show them in biological systems and how a self could emerge from that. 
%     \item We show that the mind also has this structure. 
%     \item We hypothesis that imaginaria is structured in the same way. 
% \end{itemize}

\subsection{\orange{Main Contributions}}
The main contributions of this thesis are: 
\begin{itemize}
    \item I argue that the same multi-scale hierarchies that govern biological organization can be applied to the conceptual realm
    \item I develop a method of bayesian program synthesis, using GFlowNet
    \item I present the philosophical ramifications
\end{itemize}





% Have a running example throughout the text.
% check all [sources] etc.
% question: should i put prompts to gpt for all capabilities in the intro? (abduction, analogies, etc.)
% pseudocode from other paper?
% number all equations?
% check that forward policy has \phi everywehere and generative model \theta
% also check notation \pi(s_t+1 ...) or \pi(z,...)
% change replay_prob to greek letter?
% put parts in GPT and ask for improvements
% how is the transformer decoder different from the forward logits? aka. why do we need both and not just train the transformer? sys1,2? -> ask GPT
