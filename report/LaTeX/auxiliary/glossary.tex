\newglossaryentry{homeostasis}{
    name=Homeostasis,
    description={The regulatory process by which an organism or system maintains stability while adjusting to changing external conditions.}
}

\newglossaryentry{allostasis}{
    name=Allostasis,
    description={The process by which the body achieves stability through physiological or behavioral change in response to external or internal stressors.}
}

\newglossaryentry{operational closure}{
    name=Operational Closure,
    description={Operational closure refers to a system's operations that are functionally closed, meaning that the operations are determined by the structure of the system itself and not by its environment. In the context of cognitive systems, this means that the system's cognitive processes are primarily determined by its internal states and structures rather than direct external influences.}
}

\newglossaryentry{organizational closure}{
    name=Organizational Closure,
    description={A concept in systems theory where a system is organizationally closed if its organization is maintained over time and determines the system's interactions with its environment.}
}

\newglossaryentry{intentional stance}{
    name=Intentional Stance,
    description={A cognitive strategy to predict and explain behavior by attributing beliefs, desires, and intentions to the agent, regardless of whether the agent is a human, animal, artifact, or natural phenomenon.}
}

\newglossaryentry{grounding}{
    name=Grounding,
    description={The process of linking abstract concepts or representations to concrete experiences, perceptions, or actions.}
}

\newglossaryentry{category boundary effect}{
    name=Category Boundary Effect,
    description={A cognitive bias wherein perceivers show greater sensitivity to differences that cross a categorical boundary than to equivalent differences within a category.}
}

\newglossaryentry{poverty of the stimulus}{
    name=Poverty of the Stimulus,
    description={The argument from linguistics that children do not receive enough data from their linguistic environment to infer the complex rules of a language, suggesting an innate capability for language acquisition.}
}

\newglossaryentry{structuralism}{
    name=Structuralism,
    description={A method of interpretation and analysis of aspects of human cognition, behavior, culture, and experience that focuses on relationships of contrast between elements in a conceptual system.}
}

\newglossaryentry{post-structuralism}{
    name=Post-structuralism,
    description={A theoretical framework that critiques and extends structuralism, arguing that structures are not universally applicable and are instead specific to particular times and places.}
}

\newglossaryentry{simulation}{
    name=Simulation,
    description={The imitation of a situation or process in a model or virtual environment.}
}

\newglossaryentry{simulacrum}{
    name=Simulacrum,
    description={A representation or imitation of a person or thing, often without the substance or qualities of the original.}
}

\newglossaryentry{semiotics}{
    name=Semiotics,
    description={The study of signs and symbols and their use or interpretation.}
}

\newglossaryentry{ideomotor theory}{
    name=Ideomotor Theory,
    description={The idea that simply thinking about a movement can cause a reflexive muscular response.}
}

\newglossaryentry{agency}{
    name=Agency,
    description={The capacity of individuals to act independently and make choices or decisions.}
}

\newglossaryentry{autopoiesis}{
    name=Autopoiesis,
    description={A system's ability to reproduce and maintain itself by constantly regenerating its components in response to changes in its environment.}
}

\newglossaryentry{mary's room}{
    name=Mary's Room,
    description={A thought experiment in philosophy of mind which argues that there are non-physical properties and truths about consciousness that can't be grasped by physical facts alone.}
}

\newglossaryentry{reverse engineering}{
    name=Reverse Engineering,
    description={The process of analyzing a subject system to identify the system's components and their interrelationships, with the aim of recreating the system without copying it.}
}

\newglossaryentry{generative model}{
    name=Generative Model,
    description={In machine learning, a model that can generate new samples that are similar to, but not identical to, the training data.}
}

\newglossaryentry{constitutive autonomy}{
    name=Constitutive Autonomy,
    description={The ability of a system to maintain and modify its own organizational structure.}
}

\newglossaryentry{behavioural autonomy}{
    name=Behavioural Autonomy,
    description={The capability of a system to act in its environment based on its own internal rules and processes without external control.}
}

\newglossaryentry{fixed-action patterns}{
    name=Fixed-action Patterns,
    description={Innate behavioral responses to specific stimuli that are characteristic of a species and are often performed in a similar manner by all members of the species.}
}

\newglossaryentry{reflex-reactive-intuition-reasoning}{
    name=Reflex vs Reactive vs Intuition vs Reasoning,
    description={A classification of behavioral responses ranging from automatic reflexes, reactive responses based on learned associations, intuitive judgments that come quickly without conscious deliberation, to reasoning that involves conscious, logical thought processes.}
}

\newglossaryentry{effective action}{
    name=Effective Action,
    description={Action that brings about the desired outcome or achieves its intended purpose.}
}

\newglossaryentry{circular causality}{
    name=Circular Causality,
    description={Circular causality refers to the reciprocal and interdependent relationship between perception and action in cognitive agents, where each influences and is influenced by the other.
    }
}

\newglossaryentry{perception action cycle}{
    name=Perception-Action Cycle,
    description={}
}

\newglossaryentry{cognitive light cone}{
    name=cognitive light cone,
    description={}
}

\newglossaryentry{ontogenetic}{
    name=,
    description={scale of agent}
}

\newglossaryentry{epigenetic}{
    name=,
    description={scale of cells}
}

\newglossaryentry{phylogenetic}{
    name=,
    description={scale of species}
}

\newglossaryentry{oocyte}{
    name=,
    description={}
}

\newglossaryentry{blastoderm}{
    name=,
    description={}
}

\newglossaryentry{embryo}{
    name=,
    description={}
}

\newglossaryentry{morphospace}{
    name=,
    description={}
}

\newglossaryentry{embodiment}{
    name=,
    description={}
}

\newglossaryentry{constructivism}{
    name=,
    description={}
}

\newglossaryentry{counterfactual states}{
    name=,
    description={}
}

\newglossaryentry{prototype models}{
    name=,
    description={}
}


\newglossaryentry{exemplar models}{
    name=,
    description={}
}

\newglossaryentry{domain specific language}{
    name=Domain Specific Language,
    description={}
}

\newglossaryentry{abstract syntax tree}{
    name=,
    description={}
}

\newglossaryentry{semantic grammar}{
    name=,
    description={}
}


\newglossaryentry{Context-Free Grammars}{
    name=Context-Free Grammars,
    description={}
}
% \newglossaryentry{}{
%     name=,
%     description={}
% }
% \newglossaryentry{}{
%     name=,
%     description={}
% }
% \newglossaryentry{}{
%     name=,
%     description={}
% }


% context free grammar
% probabilistic cfg




% Intentional Stance: The "intentional stance" is a concept introduced by the philosopher Daniel Dennett. When taking the intentional stance towards an entity (be it a human, animal, or even a machine), one interprets the behavior of the entity in terms of beliefs, desires, intentions, and other mental states. By attributing beliefs and desires to the entity, we can predict or explain its actions. Dennett contrasts this stance with two others: the "design stance" (predicting behavior based on the known purpose of the entity) and the "physical stance" (predicting behavior based on physical laws). The intentional stance is particularly useful when dealing with complex systems where the design or physical stance would be impractical.

% Functionalism: Functionalism is a philosophy of mind that proposes mental states are defined by their causal roles rather than by their physical constituents. In other words, mental states are defined by what they do and how they interact with other states and inputs/outputs, rather than what they are made of. This perspective allows for the possibility of multiple realizability, where the same mental state could be realized in different physical systems, as long as they play the same functional role.

% Computationalism: Computationalism is the hypothesis that cognition (or the mind) is a type of computation. In essence, the mind operates by processing information, akin to how a computer processes data. This perspective is often associated with the metaphor of the mind as software and the brain as hardware. It should be noted that while computationalism assumes a functional organization of the mind (in the sense that mental processes are described in terms of transformations of informational states), it is not identical to functionalism as it commits to a more specific kind of functional organization - one that is computational.

% Functional Computationalism: Functional computationalism can be seen as a specific kind of functionalism that assumes a computational perspective on the mind. It suggests that mental states are both defined by their causal roles (as per functionalism) and are computational in nature (as per computationalism). In other words, mental states are not only defined by their causal relations with other states and inputs/outputs, but these causal relations are specifically computational – they involve the transformation and manipulation of information.
%(Could be analog etc.)

% Diachronic Emergence: This term is used to describe emergent phenomena that occur over time. The term "diachronic" comes from the Greek words "dia," meaning "through," and "chronos," meaning "time." In this context, diachronic emergence refers to the idea that new properties or behaviors can emerge from a system as it evolves over time. For example, the process of natural selection leading to the evolution of new species is a form of diachronic emergence.

% Strong Asynchronic Emergence: This term is used to describe emergent phenomena that are not reducible to or predictable from the properties of their constituent parts, even in principle. The term "asynchronic" refers to the idea that these emergent properties exist at the same time as the system from which they emerge. In this context, strong asynchronic emergence refers to the idea that a system can have properties or behaviors that are fundamentally new and irreducible to the properties or behaviors of its parts. For example, consciousness might be considered a strongly asynchronically emergent property of certain complex physical systems, like brains.


% Ask GPT to add words and glossary items 
