% \section{Introduction}
% The human mind, in its complexity and adaptability, has long been a subject of fascination and inquiry for scientists and philosophers alike. The concept of 'self', a unique and individual consciousness that arises from cognitive processes, is a cornerstone of this exploration. The emergence of the 'self' is a phenomenon that remains elusive, despite the significant strides made in cognitive science and neuroscience. This thesis aims to delve into the intricacies of cognitive processes, specifically focusing on model-based and model-free approaches, and their potential role in the emergence of the 'self'.

% Cognitive processes are the mechanisms through which we perceive, think, remember, and understand the world around us. Two primary paradigms have been proposed to explain these processes: model-based and model-free approaches. Model-based cognition posits that our minds construct internal models of the world, which we use to predict and interpret future events. This approach suggests a dynamic and adaptable cognitive process, where our understanding of the world is constantly updated based on new experiences and information.

% On the other hand, model-free cognition proposes that our responses to the world are based on learned associations and habits, rather than internal models. This approach suggests a more static cognitive process, where behavior is driven by reinforcement learning and the repetition of actions that have previously led to rewarding outcomes.

% The dichotomy between these two approaches raises intriguing questions about the nature of the 'self'. Does the 'self' emerge from the dynamic predictions and interpretations of the model-based approach, or does it arise from the static associations and habits of the model-free approach? Or perhaps, is the 'self' a product of an intricate interplay between these two cognitive processes?

% This thesis will explore these questions, drawing on a wide range of research from cognitive science, neuroscience, psychology, and philosophy. It will critically analyze the strengths and weaknesses of both model-based and model-free approaches, and their implications for our understanding of the 'self'. Furthermore, it will investigate how these approaches can be integrated into a comprehensive framework for understanding the emergence of the 'self'.

% The exploration of the 'self' through the lens of model-based and model-free cognition offers a novel perspective on one of the most profound questions of human existence. By unraveling the cognitive processes that underpin our sense of 'self', we can gain a deeper understanding of our minds, our behavior, and our place in the world. This thesis aims to contribute to this understanding, and to stimulate further research and discussion in this fascinating field.


% - human intelligence seems to find the essence of situations. we are able to quickly condense information and extrapolate 
% - is working memory like LLMs range?


% The impressive capabilities of LLMs are undeniable, however they have limitations as well as differences to how humans understand the world. 


% The aim of the thesis is to investigate possible approaches of cognition and human thought. 
% What's so cool about human thought?
% We are able to imagine, plan, reason, consider counterfactuals, induction, abduction, deduction, causal relationships,  etc. 
