\newglossaryentry{latex}
{
    name=latex,
    description={LaTeX (short for Lamport TeX) is a document preparation system. The user has to think about only the content to put in the document and the software will take care of the formatting}
}

\newglossaryentry{glsy}
{
    name=glossary,
    description={Acronyms and terms which are generally unknown or new to common readers}
}


% \newglossaryentry{}
% {
%     name=,
%     description={}
% }

% Homeostasis
% Allostasis
% Operational Closure
% Organizational Closure
% Intentional Stance
% Grounding
% Category Boundary effect
% Poverty of the Stimuluts
% Structuralism
% Post-structuralism
% Simulation 
% Simulacrum
% Semiotics
% Ideomotor theory
% Agency
% Autopoiesis
% Mary's room
% Reverse engineering
% Generative model
% Constitutive autonomy
% Behavioural autonomy
% Fixed-action patterns
% Reflex vs reactive vs intuition vs reasoning

\newglossaryentry{operational closure}{
    name=Operational Closure,
    description={Operational closure refers to a system's operations that are functionally closed, meaning that the operations are determined by the structure of the system itself and not by its environment. In the context of cognitive systems, this means that the system's cognitive processes are primarily determined by its internal states and structures rather than direct external influences. This doesn't mean the system is isolated from its environment, but rather its interactions are mediated by its own structural dynamics.}
}

\newglossaryentry{organizational closure}{
    name=Organisational Closure,
    description={This term is closely related to \gls{operational closure} but emphasizes the organization of processes within a system. In systems theory and especially in the work of Varela and others on autopoiesis, a system displays organizational closure when its processes are recursively dependent upon each other for their generation and realization. This recursive dependency ensures that the system maintains its organization across time, even as the specific components or processes might change or be replaced. In biological terms, a cell maintains its identity and function even as individual molecules within the cell are continuously replaced.}
}

% Intentional Stance: The "intentional stance" is a concept introduced by the philosopher Daniel Dennett. When taking the intentional stance towards an entity (be it a human, animal, or even a machine), one interprets the behavior of the entity in terms of beliefs, desires, intentions, and other mental states. By attributing beliefs and desires to the entity, we can predict or explain its actions. Dennett contrasts this stance with two others: the "design stance" (predicting behavior based on the known purpose of the entity) and the "physical stance" (predicting behavior based on physical laws). The intentional stance is particularly useful when dealing with complex systems where the design or physical stance would be impractical.

% Functionalism: Functionalism is a philosophy of mind that proposes mental states are defined by their causal roles rather than by their physical constituents. In other words, mental states are defined by what they do and how they interact with other states and inputs/outputs, rather than what they are made of. This perspective allows for the possibility of multiple realizability, where the same mental state could be realized in different physical systems, as long as they play the same functional role.

% Computationalism: Computationalism is the hypothesis that cognition (or the mind) is a type of computation. In essence, the mind operates by processing information, akin to how a computer processes data. This perspective is often associated with the metaphor of the mind as software and the brain as hardware. It should be noted that while computationalism assumes a functional organization of the mind (in the sense that mental processes are described in terms of transformations of informational states), it is not identical to functionalism as it commits to a more specific kind of functional organization - one that is computational.

% Functional Computationalism: Functional computationalism can be seen as a specific kind of functionalism that assumes a computational perspective on the mind. It suggests that mental states are both defined by their causal roles (as per functionalism) and are computational in nature (as per computationalism). In other words, mental states are not only defined by their causal relations with other states and inputs/outputs, but these causal relations are specifically computational – they involve the transformation and manipulation of information.
%(Could be analog etc.)

% Diachronic Emergence: This term is used to describe emergent phenomena that occur over time. The term "diachronic" comes from the Greek words "dia," meaning "through," and "chronos," meaning "time." In this context, diachronic emergence refers to the idea that new properties or behaviors can emerge from a system as it evolves over time. For example, the process of natural selection leading to the evolution of new species is a form of diachronic emergence.

% Strong Asynchronic Emergence: This term is used to describe emergent phenomena that are not reducible to or predictable from the properties of their constituent parts, even in principle. The term "asynchronic" refers to the idea that these emergent properties exist at the same time as the system from which they emerge. In this context, strong asynchronic emergence refers to the idea that a system can have properties or behaviors that are fundamentally new and irreducible to the properties or behaviors of its parts. For example, consciousness might be considered a strongly asynchronically emergent property of certain complex physical systems, like brains.


% Ask GPT to add words and glossary items 
